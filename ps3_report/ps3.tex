\documentclass[letterpaper]{article}

%\usepackage{times}
\usepackage{epsfig}
\usepackage{graphicx}
\usepackage{amsmath}
\usepackage{amssymb}
\usepackage{subfig}

% Include other packages here, before hyperref.

% If you comment hyperref and then uncomment it, you should delete
% egpaper.aux before re-running latex.  (Or just hit 'q' on the first latex
% run, let it finish, and you should be clear).
%\usepackage[pagebackref=true,breaklinks=true,letterpaper=true,colorlinks,bookmarks=false]{hyperref}

% Pages are numbered in submission mode, and unnumbered in camera-ready
\begin{document}

%%%%%%%%% TITLE
\title{Challenge Problem 3 Solution{}}

\author{Siddharth Batra}

\maketitle


%%%%%%%%% BODY TEXT
\section{Decision Trees}
\subsection{}
\begin{align*}
 &P(y|x) = P{_{l(x)}}^y(1-P_{l(x)})^{1-y} \\ 
 &P(y|x) = \phi^y(1-\phi)^{1-y}
\end{align*}

\begin{align*}
 &\alpha(T) = \prod_{i=1}^{m} P(y^i|x^i) \\
 &\alpha(T) = \prod_{i=1}^{m}  \phi^{y^i}(1-\phi)^{1-{y^i}}  \\ \\
 &\alpha_{\delta(l)}(T) = \prod_{i=1}^{m}  \phi^{y^i}(1-\phi)^{1-{y^i}} \\
 &\alpha_{\delta(l)}(T) = \phi^{g}(1-\phi)^{m_{l}-g} \\ \\
 &\textbf{where:}\\
 &\phi = P_{l(x)} \\
 &g \text{ is total good samples}\\
 &m_{l} \text{ is total samples in } \delta(l)\\
 &(x^i,y^i) \epsilon \delta(l)  \\
\end{align*}

Taking log and setting derivate to zero

\begin{align*}
 & \frac{\partial l_{\delta(l)}(T)}{\partial \phi} = \frac{g}{\phi} - \frac{m_{l}-g}{1-\phi} = 0\\
& \phi =  P_{l(x)} = \frac{g}{m_{l}}
\end{align*}

\subsection{}

The solution is inspired by Adaboosting
and seeks to iteratively form a distribution over the samples which as the iterations
progress gives more mass to samples that are harder to classify.
\\
\\
This distribution will be used for sampling examples in the first part and the update
rule for the distribution is the solution for the second part.

\subsubsection{}


\begin{align*}
&(x^{(i)},y^{(i)}) \sim P_{W_{k}}
\end{align*}

\subsubsection{}

\begin{align*}
&W_{k+1}(j) = \frac{W_{k}(j)}{Z_{k}} \exp(-e_{b_{k}}y^{(j)}b_{k}(x^{'(j)}))\\\\
&\textbf{where:}\\
&Z_{k} \text{ is a normalization constant} \\
\end{align*}


Assuming labels +1 and -1 to make the math simpler, the above equation will
give a smaller weight (0-1) to samples that are correct (-ve exponential) and
will give a much larger weight (+ve exponential) to the mistakes while the
current error of the classifier scales the weight according to the current
performance.\\\\


\section{Error bound for 1-nearest neighbour classifier}

\subsection{}

\begin{align*}
&q(x) = p(Y=1|X=x)\\
&q(x) = \frac{p(X=x|Y=1)P(Y=1)}{P(X=x)}\\
&q(x) = \frac{p(X=x|Y=1)P(Y=1)}{P(X=x|Y=1)P(Y=1)+P(X=x|Y=0)P(Y=0)}\\
&q(x) = \frac{P_{1}(x)\theta}{P_{1}(x)\theta+P_{0}(x)(1-\theta)}\\
\end{align*}

\subsection{}

Probability of misclassification - 

\begin{align*}
&=P(X=x|Y=1)P(Y=0|X=x)+P(X=x|Y=0)P(Y=1|X=x)\\
&=P(X=x|Y=1)(1-q(x))+P(X=x|Y=0)q(x)\\
&=\frac{P(Y=1|X=x)P(X=x)}{P(Y=1)}(1-q(x))+\frac{P(Y=0|X=x)P(X=x)}{P(Y=0)}q(x)\\
&=\frac{q(x)P(X=x)(1-q(x))}{\theta}+\frac{(1-q(x))P(X=x)q(x)}{1-\theta}\\
&=\frac{q(x)(1-q(x))P(X=x)}{\theta(1-\theta)}\\
\end{align*}

\subsection{}

Probability of misclassification - 

\begin{align*}
&=P(X=x|Y=0)P(X=x'|Y=1)+P(X=x|Y=1)P(X=x'|Y=0)\\
&=\frac{(1-q(x))P(X=x)q(x')P(X=x')+q(x)P(X=x)(1-q(x'))P(X=x')}{P(Y=0)P(Y=1)}\\
&=\frac{(1-q(x))P(X=x)q(x')P(X=x')+q(x)P(X=x)(1-q(x'))P(X=x')}{\theta(1-\theta)}\\
&=P(X=x)P(X=x')\frac{(1-q(x))q(x')+q(x)(1-q(x'))}{\theta(1-\theta)}\\
\end{align*}

\subsection{}

Probability of misclassification with infinite samples - 

\begin{align*}
&=P(X=x)P(X=x')\frac{(1-q(x))q(x)+q(x)(1-q(x))}{\theta(1-\theta)}\\
&=2P(X=x)P(X=x')\frac{(1-q(x))q(x)}{\theta(1-\theta)}\\
\end{align*}

\subsection{}

\begin{align*}
&2P(X=x)P(X=x')\frac{(1-q(x))q(x)}{\theta(1-\theta)} < 2\frac{q(x)(1-q(x))P(X=x)}{\theta(1-\theta)}\\
&2P(X=x') < 2\\\\
&P(X=x') < 1\\\\
&\textbf{which is true since the problem indirectly assumes:}\\
&0 < P(X=x') < 1 \\
\end{align*}

\subsection{}

In the non-asymptotic case the number of training examples is not enough
for the label probability distributions for any test point and its nearest
neighbour to be the same. So for a test point  $x'$ the distributions of $q(x')$ and
$q(x)$ where $x$ is it's nearest neighbour are not the same, thereby invalidating the
proof under this condition.


\section{Hierarchical Clustering}

The majority of the time complexity of Hierarchical Clustering is spent in maintaining
the rule that at every time step the cluster centroids with the shortest distance must be 
merged.\\\\
Since this rule is relaxed the core idea of the solution is to use a LSH family for the
Euclidean space to assert a probablistic notion of closeness rather than the deterministic
notion that distance gives.

\subsection{}

1. Take $h$ hash functions to form a Euclidean family of functions and hash every point into
a series of buckets with width $a$.\\\\2. Iteratively or randomly chose any bucket and merge 
any two points in that bucket while removing redundant entries in the hash table.\\\\3. Hash the new cluster centroid using the $h$ hash functions and repeat.\\\\
As the book states, two points which have a distance $ d >> a$ have a small chance of being hashed
to the same bucket. More formally a Euclidean family is a $(a/2,2a,1/2,1/3)$ LSH family.\\\\
Optionally, we can apply an AND construction with $r$ rows followed by an OR construction 
with $b$ bands to amplify the family described above. 

\subsection{}

Complexity of (1) is $hn$ and complexity of (3) is $h$. Since the last two steps are 
performed at most $n$ times and the first step is only performed once the overall complexity is $hn + hn = O(hn)$.
This assumes that the complexity of (2) is negligible in comparison.\\\\ 
For the amplified case the complexity of (1) is $hrbn$ and complexity of (3) is $hrb$. The
overall complexity for this case becomes $hrbn + hrbn = O(hrbn)$

\subsection{}

The LSH Euclidean family can simply be swapped for a non-Euclidean family depending on
which non-Euclidean distance metric is appropriate. For the more interesting issue of
representing a cluster centroid, we can pick one of the points being clustered that is more
representative of the cluster and delete the hash table entry of the other point.

%\[ I = \text{an image from the set of images for class } c \]
%\[ G_i^c = \text{set of ground truth bounding boxes for class } c \text{ on image } I \]
%\[ P_i^c = \text{set of positive images for simulated sliding window } \]
%\[ N_i^c = \text{set of negative images for simulated sliding window } \]
%\[ S^c = \text{set of non-overlapping bounding box samples for image } I \]
%\[ g^{(j)} , s^{(j)} \epsilon \{ x , y , w , h \} \]
%\[ \mu_w = \text{empirical mean of widths for positive snippets } \]
%\[ \mu_h = \text{empirical mean of heights for positive snippets } \]
%\[ \Sigma_w = \text{empirical variance of widths for positive snippets}  \]
%\[ \Sigma_h = \text{empirical variance of heights for positive snippets}  \]
%\[\boldsymbol{P_i^c} = \{I({g^{(1)}}),I({g^{(2)}}),\ldots,I({g^{(n)}})\}\]
%\[\boldsymbol{N_i^c} = \{I({s^{(1)}}),I({s^{(2)}}),\ldots,I({s^{(n)}})\}\]
%\[ {s^{(j)}}(w) \sim Gaussian( \mu_w , \Sigma_w) \]
%\[ {s^{(j)}}(h) \sim Gaussian( \mu_h , \Sigma_h) \]
%\[ {s^{(j)}}(x) \sim Unif( 0 , I(w) ) \]
%\[ {s^{(j)}}(y) \sim Unif( 0 , I(h) ) \]


%-------------------------------------------------------------------------


%\begin{equation} 
%    {x_j^{(i)}} = {\arg \min}_{f \epsilon F} {||f - {d}^{(j)}||_2}^2 
%    \label{eq:fv_bw}
%\end{equation}


\end{document}
