\documentclass{article}
\usepackage{nips07submit_e,times}
\usepackage{amsmath}
\usepackage{amssymb}
%\documentstyle[nips07submit_09,times]{article}


\title{Adaptive Locality Sensitive Hashing for Recommending Twitter Followers}


\author{
Siddharth Batra \\
Department of Computer Science\\
Stanford  University\\
\texttt{sidbatra@cs.stanford.edu} \\
}

% The \author macro works with any number of authors. There are two commands
% used to separate the names and addresses of multiple authors: \And and \AND.
%
% Using \And between authors leaves it to \LaTeX{} to determine where to break
% the lines. Using \AND forces a linebreak at that point. So, if \LaTeX{}
% puts 3 of 4 authors names on the first line, and the last on the second
% line, try using \AND instead of \And before the third author name.

\newcommand{\fix}{\marginpar{FIX}}
\newcommand{\new}{\marginpar{NEW}}

\begin{document}

\maketitle

\begin{abstract}
Over the last couple of years, real-time information network Twitter has shot into
the limelight amassing hundreds of millions of users. This magnitude of users
and their tweets give insight into local and global trends but make it harder for users
to find interesting people beyond the one size fit all celebrities. The aim
of this work is to analyze a user's tweet history and find similar
users the person might be interested in. The user tweets are treated as a single
document and along with usage patterns are used to generate feature vectors. 
An adaptive approach to LSH families for the Euclidean distance is proposed for
finding nearest neighbors in this high-dimensional space and the results
are validated by treating the existing connections of a user as ground truth. As a
secondary contribution the work also provides interesting statistics about
Twitter usage patterns.
\end{abstract}

\section{Introduction}

Real-time information network Twitter \cite{twitter} allows users to share updates with their followers
and the world. This simple yet powerful mechanism allows millions of users to submit
updates and things of interest enabling Twitter to leverage data mining techniques 
for discovering global and local trends.\\\\
The sad side of this beautiful ecosystem with a large asymmetric social graph 
is that most users feel clueless about whom to connect with. Twitter does provide
a fixed directory of recommended users to follow in different categories but this 
one size fits all approach is a far cry from personalized recommendations.\\\\
The lack of recommendations of similar people creates a gap between content
creators who want relevant users to consume their content and content consumers
who want relevant content to consume. This also leads to bizarre attempts to find
connections such as offering followers for sale on eBay \cite{ebay}.\\\\
Also note that the notion of
similar does not imply that the recommended user will be exactly the same thereby
making his/her content redundant but implies that in the high-dimensional space
used to represent users some of the dimensions will be similar thereby opening the
possibility of learning more about the similar dimensions and discovering interesting
content in dissimilar dimensions.\\\\
With this motivation, the aim of this work is to present a framework for recommending
similar users based on a person's tweeting history and usage patterns. A comparison
to finding similar documents is proposed and the choice of the feature vectors
is discussed in this light. An adaptive approach to locality sensitive hashing in
Euclidean space is proposed for efficient and accurate nearest neighbor search. This is
used in conjunction with a supervised learning algorithm to enhance results. The work
is validated by using the existing social graph of users on Twitter: note that this
social graph is \textit{not} leveraged during the adaptive LSH, a part of it is used
as a training set for the supervised learning algorithm and the rest of is used
as ground truth for the entire algorithm. As a secondary contribution a set of
usage statistics are also presented about Twitter users that can be leveraged while
designing feature vectors for various applications.


\section{Approach}

This section delineates the design of the various parts of the framework
and motivates some of the design choices.

\subsection{Users as Documents}

\begin{center}
    "Pelosi Signals Goal of Passing Health Measure Before Obama Trip http://bit.ly/9OxAs7"
\end{center}

The above text shows an example of a user tweet. Since by itself the text in
a tweet doesn't offer enough information, all the user's tweets are combined together into a 
document to enable more robust features. The definition of a user document $D_u$ is more formally illustrated in
equation \ref{eq:doc}. Notice the use of summation rather than a union, this is done to preserve the
word distribution.

\begin{center}
\begin{equation}
D_u = \sum_{i=1}^{n_u}T_{u}^{i}
\label{eq:doc}
\end{equation}
\end{center}

Given enough of a user's tweet history this formulation allows feature vectors to be built
on top of standard text features used for documents.


\subsection{Features}

\label{TODO:add not as to why at and hash are not useful support with data}

Before any algorithm for identifying nearest neighbors can be applied the raw data
must be converted to a meaningful feature vector. The raw data available is
a user's document $D_u$ and date times for every tweet in the document.

\subsubsection{Text features using $D_u$}

The data is cleaned by removing special characters, stop words, URLs etc. since they aren't 
discriminative features and by applying stemming. Next, instead of trying to
ascertain the top topics of a user's tweets which is essentially a hard filtering / prioritization
of the available words, a soft estimate of $P(w|u)$ is used as shown in equation \ref{eq:idf}.

\begin{center}
\begin{equation}
P(w|u) = TF_{u}^w = \frac{n_{u}^w}{\sum_{i}^{}n_{u}^i}
\label{eq:idf}
\end{equation}
\end{center}

Note the use of summation in the denominator rather a maximum as prescribed in the textbook.
This form maintains a more probabilistic interpretation of the TF which is lost by using the
max. The more robust TF.IDF metric was not used for logical reasons illustrated in the
sections to come. Thus, equation \ref{eq:idf} is used to assign a probability of occurrence to every
word $w$ depending on which user document it is found in. Although \textit{not}
mathematically equivalent, this can be thought of as the probability of a user $u$ using the word $w$ in
a tweet.\\\\

References to other users and hash-tags are treated as text features. Although hash tags
are meaningful but only 21\% of users utilize them and inorder to not skew features towards
a particular set of users using them as text features seems fair. For user
references this approach can allow recommendations of users who may have a strong connection in
common e.g. both are big fans of @ycombinator.

\subsubsection{Date time features}

The rough idea is that people who are similar will tend to tweet at similar times
of the day and similar days of the week. Its a debatable feature to use but the formulation
of the distance measure will compensate for this and give a high weight to the stronger
text features. This translates to a seven dimensional feature using radial basis functions
centered at 12:00PM on every day allowing for a softer measure of time periods rather
than a hard discretization. These features are more formally illustrated in equation
\ref{eq:rbf}.

\begin{center}
\begin{align}
\label{eq:rbf}
&U(dt) = \bigcup_{i=1}^{7}\exp(-{||c_{i}-dt||}_2^2) \\
&U(dt) \epsilon \mathbb{R}^7
\end{align}
\end{center}


\subsubsection{Feature vector}

The probabilities of the words in $D_u$ and the date time features
together form the feature vector representation of the user. If we consider the
unique number of words that can occur across all documents of all users then
each user vector will be very high-dimensional. Of-course, this vector will also
be very sparse since the probability mass is distributed in a select few words
. Thus, for computational and storage
efficiency the implementation only stores the words in $D_u$ their probabilities
computed using equation \ref{eq:idf} and the date time features from equation 
\ref{eq:rbf}. 

\subsection{Nearest Neighbor Search}

This section presents the weighted Euclidean distance measure and the 
efficient adaptive LSH technique used for finding nearest neighbors or
similar users in this very high-dimensional space of users that we have thus
far represented sparsely.

\subsubsection{Distance measure}

Since all the values are between 0 and 1, we can use any distance measure in the Euclidean
space for quantifying closeness between two user feature vectors.
The L2 norm is chosen for its exponential nature of weighting i.e. similar users
get very low scores and dissimilar users get a very high score. It is interesting
to note that all the columns of the feature vectors may not be of equal importance, indeed
we claimed previously that the day time features are not as important as the stronger
text ones and this notion can be achieved by introducing weights as shown in
equation \ref{eq:dist}.

\begin{center}
\begin{align}
\label{eq:dist}
&dist(U_a,U_b) = \sum_{i=1}^{|U|}w_{i}^{a}||U_a(i)-U_b(i)||_{2}^2 \\
&dist(U_a,U_b) \neq dist(U_b,U_a) \\
\end{align}
\end{center}

Note that the use of $w$ in equation \ref{eq:dist} is very similar to the notion of
variance in the Mahalanobis distance \cite{mahal}. So $w$ can be thought of as a vector
representing the variance of each dimension in this space.

\subsubsection{Supervised learning to estimate $w$}

Since this work explores personalized user recommendations, it seems natural to think of 
personalizing the weight vector $w$ for each user. The idea is to see which features are
more important for each user in finding people similar to them. The Twitter API \cite{api}
is used to find some of the user's connections amongst the 17,069,981 users available in the dataset.
The feature vectors for the user's connections serve as positive examples and other users
are randomly sampled to form a negative set. \\\\
The Kullback Leibler divergence \cite{kl} is then used
as a criterion to rank the features using an independent evaluation for binary classification.
The weights $w$ then becomes the normalized value of the KL divergence. In the ideal 
computationally unbounded world, we would be free to compute a different weight for every
feature but this requires an unrealistic amount of training data. Thus, the weights were
used primarily to control the relative weighting of the text and date time features by giving
all the text features the same weight. The results section presents more details, 
but irrespective of current API limitations it is an interesting idea to personalize $w$.

\subsubsection{Adaptive LSH}

The two main choices for finding nearest neighbors in an enormous high-dimensional space
of 17,069,981 users and 476,553,560 tweets are clustering and hashing. Clustering users
is perhaps a vague approximation for nearest neighbors since clusters can be large
and ill-defined and they also can't support asymmetric distances. Plus, it is computationally
quite expensive since it takes several passes over the whole set. The process of repeatedly
testing for which clusters to merge make it inefficient even in a map-reduce setting. \\\\
Hashing techniques are a better choice in this light and the LSH family for Euclidean
distances seems ideal. There are two main caveats for using this technique as is. First, 
for the hashing to work an index is needed for every feature. Given our efficient sparse
representation this will mean having an enormous look-up table to look up indicies of the
text features - this can get inefficient very quickly. Second, and more importantly it
will be hard to introduce personalized weights for every user during the hashing stage -
a notion of different bucket sizes for different features can perhaps be attempted but it
still suffers from the former caveat.\\\\
With these caveats in mind an adaptive LSH technique for the Euclidean family is proposed.
Adaptive implies that the hashing will be able to accommodate personalization by placing 
users in adaptive buckets using custom hash functions for each user.\\\\
The core idea is to create buckets on the fly by sampling the distribution of words of
each user. Buckets comprise of singles,pairs and triplets of words. Equation \ref{eq:buckets}
formalizes the probability of a user getting placed in an adaptive bucket.

\begin{center}
\begin{align}
\label{eq:buckets}
&P(B|u) = \frac{|B|}{Z}\prod_{b=1}^{|B|}P(w_b|u)P(w_b) \\
&|B| \epsilon Unif(3) \\
&B = \{w_1...w_{|B|}\}
\end{align}
\end{center}

The prior probability $P(w)$ can be computed using the inverse frequency metric and helps in
down-weighting buckets formed by commonly occurring words. Equation \ref{eq:buckets} formalizes
the probability $P(B|u)$ of a bucket $B$ of size $|B|$ for a particular user $u$. Note that
the $|B|$ in the numerator gives more weight to buckets with more words. $Z$ is a normalization constant.\\\\
The bucket then serves as a key in the map-reduce framework and the idea is that two users fall in
the same bucket only if their respective distributions had a high enough probability of a set 
of words that were uncommon enough - see equation \ref{eq:userbuck}. Note that this soft method of buckets enables serendipitous matching
of similar users based on their interests in a subset of dimensions.\\\\
After the users have been adaptively hashed, all the users within a bucket are compared using their full
feature vectors by the asymmetric distance measure in equation \ref{eq:dist} and matches below a threshold
are declared as recommended users. 

\begin{center}
\begin{align}
\label{eq:userbuck}
&P(u_1,u_2|B) = \frac{1}{Z}P(B|u_1)P(B|u_2)
\end{align}
\end{center}

To sum up, the probabilistic approach to adaptive LSH enables an efficient yet accurate 
method of finding nearest neighbors in this high-dimensional space. It is interesting to note
that this framework preserves the asymmetric notion of distance (i.e. user A might be recommended
to user B but not vice versa) which is characteristic of Twitter but is hard to accomplish
efficiently using LSH or clustering in their existing forms.

\section{Experiments}

The Twitter dataset provided by the course staff contains 
476,553,560 tweets from 17,069,981 users. All the experiments were carried out
on the Amazon Elastic MapReduce platform using HQL on top of Hadoop. The custom
MapReduce scripts were written in Ruby.\\\\
The aim to run and validate all the theoretical experiments proposed for adaptive
LSH on the entire available Twitter dataset required extremely efficient Ruby
scripts and loads of HighCPU EC2 machines (20-50). The entire set of experiments
on the entire dataset take $~$1.5 days to run.\\\\
Table \ref{tab:mapreduce} shows the schema of different tables that were created
using HQL and populated with Ruby mappers-reducers. There were intermediate tables to hold
the results from the map tasks but they are not as interesting.

  \begin{table}
    \begin{center}
     \begin{tabular}{ | l | l | }
     \hline
     \textbf{Schema} & \textbf{Comment}   \\ \hline
     raw text & raw file format with T,U,D on separate lines.  \\ \hline
     user,tweets,datetime & combined user tweets and usage patterns. 1MR.  \\ \hline
     user,$D_u$, $P(W|u)$,datetime RBF, & feature vectors for every user. 1M.  \\ \hline
     bucket,Users,[$D_u$],[$P(W|u)$],[datetime RBF], & all the users falling into a bucket. 1MR.  \\ \hline
     user,[users],total,[words],[distances], & final user recommendations. 1MR.  \\ \hline
     \end{tabular}
     \caption{Map reduce stages for computing user buckets.}
     \label{tab:mapreduce}
     \end{center}
   \end{table}
 
Table \ref{tab:buckets} presents some interesting user recommendations. The keywords columns 
shows words that have a high joint distribution in both user documents. For the row with multiple
recommendations it is not necessary that all the keywords will have a high joint distribution but
in particular accounting case presented they simply happen to.
 
\begin{table}
    \begin{center}
     \begin{tabular}{ | l | p{4cm} | p{7cm} | }
     \hline
     \textbf{User} & \textbf{Recommendations} & \textbf{Keywords}   \\ \hline
     acct\_jobsinhk & ir35accountant denaccounting accountancyage & accredited accountants manager limited company contract  \\ \hline
     aceball & ebookkeeper & tickets ballpark game financial credit  \\ \hline
     adriblue22 & rachaelsmart & life work planned feel school alarm \\ \hline
     aloredelam & silhouettefest & concert music festival video merci clip listening france  \\ \hline
     blackrain69 & caribworldnews & headlines morning latest faces mocking caribbeans  \\ \hline
     cradlecatholic & carenetmilwauke & ultrasound weary posted feeling bible shalt good kill time lose reap  \\ \hline
     \end{tabular}
     \caption{Interesting user recommendations.}
     \label{tab:buckets}
     \end{center}
\end{table}

An unexpected application that emerged from this work was SPAM detection on Twitter. As soon
as it became popular Twitter was bombarded with emarketers, porn sites etc. Experiments showed
that at a very low distance threshold in the nearest neighbor search, a lot of SPAM accounts were
exposed due to repeated tweets which had the same content over and over again. Table \ref{tab:spam}
presents these results.

\begin{table}
    \begin{center}
     \begin{tabular}{ | l | p{4cm}  | }
     \hline
     \textbf{Spam accounts} & \textbf{Keywords}   \\ \hline
     surrealbrian2 surrealgirls & great exposure business girls  \\ \hline
     jasonbarczewski jasonbart & earn money online retweet plz  \\ \hline
     daveperryonline davep\_etools davep\_48 &  manchester offer portugal cris  \\ \hline
     blackberryking blackberrynews & blackberry storm curve review   \\ \hline
     \end{tabular}
     \caption{SPAM detection results.}
     \label{tab:spam}
     \end{center}
\end{table}


Table \ref{tab:results} presents some interesting results about adaptive LSH and the overall
dataset. The recommendation accuracy was computed by treating the existing connections of a 
user as ground truth and looking for \% of recommended users that were existing connections.
Its worth a mention that the Twitter API is extremely restrictive and allows only 150 calls
per hour (sometimes less). This coupled with the fact that the dataset has user screen names
where-as the API returns IDs meaning that an additional call is needed for every recommended
user. This makes it nearly impossible to run validation experiments on a large set and hence only \textit{this}
particular result is validated on a smaller set (rest use the full set). Also note that is the lower bound of accuracy
since there is no metric (apart from user testing) to judge whether a non-existent connection
will be a good recommendation or not.

\begin{table}
    \begin{center}
     \begin{tabular}{ | l | p{4cm}  | }
     \hline
     \textbf{Result} & \textbf{Statistic}   \\ \hline
     Total users & 17,069,981   \\ \hline
     Total tweets & 476,553,560   \\ \hline
     Most active period &  12AM-6AM \\ \hline
     Most active day &  Monday \\ \hline
     \% of users using \# & 21\%  \\ \hline
     \% of users using @ & 46\%   \\ \hline
     Recommendations per user  &  32  \\ \hline
     Recommendation accuracy  & $\geq$ 8.1\%  \\ \hline
     Top 10 words  & love good time will  today going work great night check\\ \hline
     \end{tabular}
     \caption{Overall results.}
     \label{tab:results}
     \end{center}
\end{table}

\section{Conclusion}

The real-time information sharing network has amassed millions of users
and does a great job of discovering local and global trends. 
This paper attempts to address the problem of users not being able to find
relevant people to follow apart from the one size fit all celebrities. A
framework for implementing nearest neighbor search using asymmetric distances
that mirror Twitter's asymmetric social graph is proposed by using 
adaptive LSH. A proposal to learn the weight vector $w$ used in
the Euclidean distance measure is also presented. The framework is 
validated on a ~470 million tweet dataset and a lower bound on the recommendation
accuracy is produced by treating the existing Twitter connections as ground truth.
The recommendation accuracy is found to be $\geq$ 8.1\% and an average of 32 recommendations
are provided per user. As a secondary contribution some interesting statistics
about Twitter users are presented that can be leveraged while designing 
feature vectors for various applications.


\bibliography{report.bbl}


\end{document}
























